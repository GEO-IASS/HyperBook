\documentclass[10pt]{book}
\paperwidth=6in
\paperheight=9in

\usepackage{graphicx,color,fancyhdr}
\usepackage{hyperref}
\hypersetup{colorlinks=true, linkcolor=blue, filecolor=blue, urlcolor=blue}
\usepackage{moreverb}
% moreverb is in tetex-extra
\usepackage[left=0.625in,top=1in,right=0.5in,bottom=1in]{geometry}
% \pagestyle{empty}

%Beautifying code
\usepackage{minted}

% Define pagestyle
\pagestyle{fancy}
\fancyhf{}
\renewcommand{\chaptermark}[1]{\markboth{ \emph{#1}}{}}
\fancyhead[LO]{}
\fancyhead[RE]{\leftmark}
\fancyfoot[LE,RO]{\thepage}

% \sloppy

% Clean after an itemize
\usepackage{enumitem}
\setlist[itemize]{topsep=0pt,after=\newline}

% Define a tab conscious verbatim for python code
\def\sourcetabsize{4}
\newenvironment{sourcestyle}{\begin{scriptsize}}{\end{scriptsize}}
\def\sourceinput#1{\par\begin{sourcestyle}\verbatimtabinput[\sourcetabsize]{#1}\end{sourcestyle}\par}


% \makeatletter
% \g@addto@macro\@verbatim\footnotesize
% \makeatother
\makeatletter
\g@addto@macro\@verbatim\scriptsize
\makeatother



\begin{document}
\begin{titlepage}
\begin{center}
\ \\
\ \\
\ \\
\ \\
%
\ \\
\ \\
\ \\ % force an empty line
% textsc is small caps
\textsc{\Large
Hyper Spectral For All\\
}
\ \\
\ \\
\ \\
\ \\
\ \\
\ \\
\ \\
\ \\
\ \\
\ \\
% \includegraphics[scale=0.25]{fullreso.png}\\
\ \\
\textsc{\Large
\\
\ \\}
\ \\
\ \\
\ \\
\ \\
\ \\
\ \\
%
\vfill % fill vertical space
% split the page into two columns
% left column
\begin{minipage}{0.4\textwidth}
\begin{flushleft} \large
\end{flushleft}
\end{minipage}
%
% right column
\begin{minipage}{0.4\textwidth}
\begin{flushright}
\ \\
\ \\
\emph{Dr. Yann H. Chemin\\Dr. Margherita Di Leo}

% ensure vertical alignment with LHS
\footnotesize{\ }
\end{flushright}
\end{minipage}
%
% Bottom of the page
% \today is the compilation date
% {\large \today}
%
\end{center}
\end{titlepage}
\begin{flushleft}
 \textbf{GIS programming}
\end{flushleft}
First Edition\newline\linebreak
Dr. Yann H. Chemin\\Dr. Margherita Di Leo\newline\linebreak
ISBN  XXX-X-XXX-XXXXX-X\newline\linebreak
Public Domain\newline
\newpage
\begin{center}
 \textbf{Foreword}
\end{center}
This textbook aims at expanding basics of Hyper Spectral Science, application and programmming.
It should be taken as an overview more than a thorough material, and by no mean dealing with all of the subject. 
\newline\linebreak
After going through this book, the reader will be able to have a basic knowledge of the technology available
for Hyper Spectral data programming, and a good practical hand on most common ways to investigate them.
\newline\linebreak
Greetings and good luck,\newline
Yann Chemin\\Margherita Di Leo\newline\linebreak

\tableofcontents

\newpage

%%%%%%%%%%

\chapter{Introduction}
In a very short time (relative to Earth's age), the modern human civilization has conquered its neighbouring space with
probes, satellites, and vehicles carrying humans for exploration. Observing platforms (handhelds, 
terrestrial, airborne or space-borne) are found in a large range of observational situation. 
Starting in our hands, circumventing our inner atmosphere to its boundary, in low earth orbit up to geostationary orbit,
up to other planets and celestial bodies.\newline\linebreak
Remote Sensing is now ubiquitous and increasingly in public access on planet Earth, and thus enters a revolutionary age.
Earth monitoring satellites permit detailed, descriptive, quantitative, holistic, standardized, global evaluation
of the state of the Earth skin in a manner that our actual Earthen civilization has never been able to before.\newline\linebreak
On Earth, numerous remote sensing datasets are available, a good start could be the Reverb website 
(\href{http://reverb.echo.nasa.gov/reverb/}{reverb.echo.nasa.gov/reverb/}). Outside of this Earth, remote sensing missions 
within our solar system have largely increased and with their full datasets available online 
(\href{http://pds.nasa.gov}{pds.nasa.gov}).\newline\linebreak
In this book, Python is used to provide ready-to-go GIS programming examples. Python is a simple, expressive \& clear 
high-level programming language, it is used here to lessen the learning curve and for its readability. It is found at
(\href{http://www.python.org}{www.python.org}). It is also called by some a scripting language,
because it does not need explicit compilation into an application binary file to be used. Throughout this book, a number of
tools using bindings to Python are used. The main one is the Python-GDAL/OGR binding 
(\href{http://pypi.python.org/pypi/GDAL/}{pypi.python.org/pypi/GDAL/}), which are a number of tools for programming
and manipulating geospatial raster and vector data.\newline
\newpage

%%%%%%%%%%

\chapter{Chapter1}
\section{Introduction}
\subsection{vcvcvcvcvcvv}
\subsection{cccvcvccvcvs}


\noindent An example can be (from Merchich/Nord Maroc to WGS84 lat/long):

\begin{minted}[frame=single,linenos,mathescape,fontsize=\footnotesize]{sh}
ogr2ogr -s_srs EPSG:26191 -t_srs EPSG:4326 outfileWgs84.shp infileMerchich.shp
\end{minted}

\subsection{Loading required libraries in Python}
Few things are required to be able to let Python understand the instructions in the following examples. 
They are {\it osr} and {\it ogr}. {\it osr} is a library to manipulate spatial references systems (projections).
{\it ogr} is a library to manipulate geometry entities and their vector formats containers.

\begin{minted}[frame=single,linenos,mathescape,fontsize=\footnotesize]{py}
#!/usr/bin/env python
#import required libraries

from osgeo import osr
from osgeo import ogr
\end{minted}

\newpage

\newpage

%%%%%%%%%%

\chapter{Final Notes}

\newpage
\section{Acknowledgements}
Planetary imagery found in this book are courtesy of NASA/JPL, vector data is courtesy of \href{http:/gadm.org}{gadm.org}
\section{License for the code in this book}
All code in this book is Public Domain, if you really need to include information about the author 
the following can be used.
\begin{minted}[fontsize=\footnotesize]{py}
###############################################################################
# Public Domain 2015-2016
# Yann Chemin <yann.chemin@gmail.com>
# Margherita Di Leo <dileomargherita@gmail.com>
###############################################################################
\end{minted}
\end{document}
